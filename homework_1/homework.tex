%This is my super simple Real Analysis Homework template

\documentclass{article}
\usepackage[utf8]{inputenc}
\usepackage[english]{babel}
\usepackage[]{amsthm} %lets us use \begin{proof}
\usepackage[]{amssymb} %gives us the character \varnothing
\usepackage[]{amsmath}
% testing branch

\title{Homework 1}
\author{Chu Hai Nam MSSV: 2370189}
\date\today
%This information doesn't actually show up on your document unless you use the maketitle command below

\begin{document}
\maketitle %This command prints the title based on information entered above

%Section and subsection automatically number unless you put the asterisk next to them.
\section*{Section 3.2}
Let $m:\mathcal{A}\rightarrow [0,\infty)$ be a set function where $\mathcal{A}$ is a $\sigma$-algebra. Assume $m$ is countably additive over countable disjoint collections of sets in $\mathcal{A}$.
%Basically, you type whatever text you want and use the $ sign to enter "math mode".
%For fancy calligraphy letters, use \mathcal{}
%Special characters are their own commands

\subsection*{Problem 1}
Given sets $A$, $B$, and $C$, if $A\subset B \textrm{ and } B \subset C$, then $A \subset C$.
\begin{proof}
Other symbols you can use for set notation are
\begin{itemize}
\item$A \supset B \supseteq C \subset D \subseteq E$. Also $\varnothing \textrm{vs} \emptyset$
\item$\cup$ and $\cup_{k=1}^\infty E_k$
\item$\cap$ and $\cap_{x \in \mathbb{N}} \{\frac{1}{\sqrt[3]{x}}\}$
\item$\bigcup$ and $\bigcap\limits_{k=0}^n$ and $\bigcap$
\item most Greek letters $\sigma \pi \theta \lambda_i e^{i\pi}$
\item $\int_0^2 ln(2)x^2sin(x) dx$
\item$\leq < \geq > = \neq$
\end{itemize}
If you want centered math on its own line, you can use a slash and square bracket.\\
\[
\left \{
\sum\limits_{k=1}^\infty l(I_k):A\subseteq \bigcup_{k=1}^\infty \{I_k\}
\right \}
\]
The left and right commands make the brackets get as big as we need them to be.
\end{proof}

\clearpage %Gives us a page break before the next section. Optional.
\subsection*{Problem 2}
Prove equation (3.19) states: \begin{equation} \lg(n!) = \Theta(n \lg n)\end{equation}
\begin{proof}
    $ $\newline
    $ $\newline
    We use Stirling's approximation for this proof \\
    With large values of n, $\Theta (\frac{1}{n}) $is very smaller than 1. \\
    So we can write n! as follows: \\
    \begin{equation}
        \begin{aligned}
        \lg (n !) & \approx \lg \left(\sqrt{2 \pi n}\left(\frac{n}{e}\right)^n\right) \\
        & =\lg (\sqrt{2 \pi n})+\lg \left(\frac{n}{e}\right)^n \\
        & =\lg \sqrt{2 \pi}+\lg \sqrt{n}+n \lg \left(\frac{n}{e}\right) \\
        & =\lg \sqrt{2 \pi}+\frac{1}{2} \lg n+n \lg n-n \lg e \\
        & =\Theta(1)+\Theta(\lg n)+\Theta(n \lg n)-\Theta(n) \\
        & =\Theta(n \lg n)
        \end{aligned}
    \end{equation}
\end{proof}

\subsection*{Problem 3}
%
\begin{proof}
%
\end{proof}

\section*{Section 2.2}
%
\subsection*{Problem 6}
Blah
\subsection*{Problem 7}
Blah
\subsection*{Problem 10}
Blah

\end{document}