%This is my super simple Real Analysis Homework template

\documentclass{article}
\usepackage[utf8]{inputenc}
\usepackage[english]{babel}
\usepackage[]{amsthm} %lets us use \begin{proof}
\usepackage[]{amssymb} %gives us the character \varnothing
\usepackage[]{amsmath}

\usepackage{authblk}
\title{Homework 1}
\author{Chu Hai Nam MSSV: 2370189}
\author{Diep Le Vy MSSV: 2370200}
\author{Huynh Huy Vu MSSV: 2370199}

\date\today
%This information doesn't actually show up on your document unless you use the maketitle command below

\begin{document}
\maketitle %This command prints the title based on information entered above

\section*{Section 3.2}
Standard notations and common functions

\subsection*{Problem 3.2.3}
Is $2^{n+1} = O(2^n)$ ? Is $2^{2n} = O(2^n)$ ?
\begin{proof}
    $ $\newline
    $ $\newline
    We determine positive constants $c$ and $n_0$ such that $0\le2^{n+1}\le c2^n$ for all $n\ge n_0$. \\
    Since $2^{n+1}=2\cdot2^n$, we can pick $c=2$ and $n_0=1$. \\
    So $2^{n+1}=O(2^n)$. \\
    $ $\newline
    We determine positive constants $c$ and $n_0$ are positive constants satisfying $0\le2^{2n}\le c2^n$ for all $n\ge n_0$. \\
    Then $2^{2n}=2^n\cdot2^n\le c2^n$, which implies that $c\ge2^n$. \\
    We would not choose the $c$, because $2^n$ becomes arbitrarily large as $n$ gets large. \\
    So, $2^{2n}\ne O(2^n)$. \\
\end{proof}

\newcommand{\Tworst}{T_{worst}(n)}
\newcommand{\Tbest}{T_{best}(n)}
\subsection*{Problem 3.2.5}
Prove that the running time of an algorithm is $ \Theta(g(n)) $ if and only if its worst-case
running time is $ O(g(n)) $ and its best-case running time is $ \Omega(g(n)) $.
\begin{proof}
    $ $\newline
    $ $\newline
    For the input size $n$ with $T(n)$ be the running time of an algorithm, 
    and let $\Tworst$ and $\Tbest$ be its worst-case running time and its best-case running time, respectively. \\
    $ $\newline
    Suppose that $T(n)=\Theta(g(n))$. \\
    For all input size $n$, there is one worst case and one best case for the algorithm 
    in terms of time required to run on these inputs. \\
    Since $T(n)$ describes the running time for \emph{all} cases, it also describes the 
    running time for the worst case and the best case. \\
    Thus, $\Tworst=\Theta(g(n))$ and $\Tbest=\Theta(g(n))$, and by Theorem 3.1 \\
    We have that $\Tworst=O(g(n))$ and $\Tbest=\Omega(g(n))$. \\
    $ $\newline
    Suppose that $\Tbest=\Omega(g(n))$ and $\Tworst=O(g(n))$. \\
    Let $c_1$ and $n_1$ be positive constants such that $0\le c_1g(n)\le\Tbest$ for all $n\ge n_1$, \\
    Let $c_2$ and $n_2$ be positive constants such that $0\le\Tworst\le c_2g(n)$ for all $n\ge n_2$. \\
    For any input size $n$, $\Tbest\le T(n)\le\Tworst$.
    Then for all $n\ge\max{n_1,n_2}$,
    \[
        0 \le c_1g(n) \le \Tbest \le T(n) \le \Tworst \le c_2g(n),
    \]
    So $T(n)=\Theta(g(n))$.
\end{proof}

\clearpage %Gives us a page break before the next section. Optional.
\subsection*{Exercise 5.2-6}
Let $A[1 \ldots n]$ be an array of $n$ distinct numbers. If $i < j$ and $A[i] > A[j]$, 
then the pair $(i, j)$ is called an inversion of $A$. 
(See Problem 2-4 on page 47 for more on inversions.) 
Suppose that the elements of $A$ form a uniform random permutation of $\{1, 2, \ldots, n\}$. 
Use indicator random variables to compute the expected number of inversions.

\begin{proof}

Let $X_{ij}$ be an indicator random variable for the event where the pair $A[i], A[j]$ for $i < j$ is inverted, i.e., $A[i] > A[j]$. 
More precisely, we define $X_{ij}$ as:
\[ X_{ij} = I\{A[i] > A[j]\} \text{ for } 1 \leq i < j \leq n \\
\]

We have $\Pr(X_{ij} = 1) = \frac{1}{2}$ because given two distinct random numbers, the probability that the first is bigger than the second is $\frac{1}{2}$. 
By Lemma 5.1, $E[X_{ij}] = \frac{1}{2}$.

Let $X$ be the random variable denoting the total number of inverted pairs in the array, so that:
\[ X = \sum_{i=1}^{n-1} \sum_{j=i+1}^{n} X_{ij}
\]

We want the expected number of inverted pairs, so we take the expectation of both sides of the above equation to obtain:
\[ E[X] = E\left[ \sum_{i=1}^{n-1} \sum_{j=i+1}^{n} X_{ij} \right]
\]

We use linearity of expectation to get:
\[ E[X] = \sum_{i=1}^{n-1} \sum_{j=i+1}^{n} E[X_{ij}]
\]

This simplifies to:
\[ E[X] = \sum_{i=1}^{n-1} \sum_{j=i+1}^{n} \frac{1}{2}
\]

\[     = \sum_{i=1}^{n-1} \frac{n - i}{2}
\]

\[     = \frac{1}{2} \sum_{i=1}^{n-1} (n - i)
\]

\[     = \frac{1}{2} \cdot \frac{n(n-1)}{2} = \frac{n(n-1)}{4}
\]

Thus, the expected number of inverted pairs is $\frac{n(n-1)}{4}$.
\end{proof}

\subsection*{Exercise 7.2-3}
%
Show that the running time of QUICKSORT is $\theta (n^2)$ when the array $A$ contains distinct elements and is sorted in decreasing order.

\begin{proof}
%

For each loop, QUICKSORT call on 2 sub-array, 1 is empty and another sub-array contain n-1 elements.

\begin{itemize}

\item 
    Suppose that PARTITION is called on a subarray $A[p:r]$ 
    whose elements are distinct and in decreasing order ($A[p] > A[r]$)
    PARTITION chooses the pivot $A[r]$, which is the smallest element.
    Every test in line 4 comes up false, so that no elements are exchanged during the execution of the for loop. 
    Before PARTITION returns, line 6 finds that $i \neq p + 1$, 
    and so it swaps the elements in $A[p]$ and $A[r]$. 
    PARTITION returns $r$ as the position of the pivot. 
    The subarray containing elements less than or equal to the pivot is empty. 
    The subarray containing elements greater than the pivot, $A[p+1:r]$, 
    has all but the pivot and is in decreasing order, 
    except that the maximum element of this subarray is in $A[r]$.

\item 
    When QUICKSORT calls PARTITION on $A[p:q-1]$, nothing changes, 
    as this subarray is empty. 
\item 
    When QUICKSORT calls PARTITION on $A[q+1:r]$, now the pivot is the greatest element in the subarray. 
    Although every test in line 4 comes up true, the indices $i$ and $j$ are always equal in line 6, 
    so that just as in the case where the pivot is the smallest element, 
    no elements are exchanged during the execution of the for loop. 
    Before PARTITION returns, line 6 finds that $i \neq r + 1$, 
    so that the swap in line 6 leaves the pivot in $A[r]$. 
    PARTITION returns $r$ as the position of the pivot. 
    Now the subarray containing elements less than or equal to the pivot has all but the pivot and is in decreasing order, 
    and the subarray containing elements greater than the pivot is empty. 
    The next call to PARTITION, therefore, is on a subarray that is in decreasing order, 
    so that it goes back to the first case above.

\end{itemize}
Therefore, each recursive call is on a subarray only one element smaller, 
giving a recurrence for the running time of $T(n) = T(n-1) + \Theta(n)$, 
so the solution is $\Theta(n^2)$.
\end{proof}

\newpage 
%
\subsection*{Excerise 7.2-5}
Suppose that the splits at every level of quicksort are in the constant proportion $\alpha$ to $\beta$, where $\alpha + \beta = 1$ and $0 < \alpha, \beta < 1$. 
Show that the minimum depth of a leaf in the recursion tree is approximately $\ln \frac{1}{\alpha} n$ and that the maximum depth is approximately $\ln \frac{1}{\beta} n$. (Don't worry about integer round-off.)
\begin{proof}
The minimum depth follows a path that always takes the smaller part of the partition—i.e., 
that multiplies the number of elements by $\alpha$. 
One level of recursion reduces the number of elements from $n$ to $\alpha n$, 
and $i$ levels of recursion reduce the number of elements to $\alpha^m n$. 
At a leaf, there is just one remaining element, 
and so at a minimum-depth leaf of depth $m$, we have $\alpha^m n = 1$. 
Thus, $\alpha^m = \frac{1}{n}$. 
Taking logarithms, we get $m \ln \alpha = \ln n$, or $m = \frac{\ln n}{\ln \alpha}$. 
(This quantity is positive because $0 < \alpha < 1$ implies that $\ln \alpha < 0.)


Similarly, the maximum-depth path corresponds to always taking the larger part of the partition, 
i.e., keeping a fraction $\beta$ of the elements each time. 
The maximum depth $M$ is reached when there is one element left, 
that is, when $\beta^M n = 1$. 
Thus, $M = \frac{\ln n}{\ln \beta}$. 
(Again, this quantity is positive because $0 < \beta < 1$ implies that $\log \beta < 0.)

All these equations are approximate because we are ignoring floors and ceilings.

\end{document}