%This is my super simple Real Analysis Homework template

\documentclass{article}
\usepackage[utf8]{inputenc}
\usepackage[english]{babel}
\usepackage[]{amsthm} %lets us use \begin{proof}
\usepackage[]{amssymb} %gives us the character \varnothing
\usepackage[]{amsmath}

\title{Homework 1}
\author{Chu Hai Nam MSSV: 2370189}
\date\today
%This information doesn't actually show up on your document unless you use the maketitle command below

\begin{document}
\maketitle %This command prints the title based on information entered above

\section*{Section 3.2}
Standard notations and common functions

\subsection*{Problem 3.2.3}
\begin{itemize}
    \item Prove equation: \begin{equation} \lg(n!) = \Theta(n \lg n)\end{equation}
    \begin{proof}
        $ $\newline
        $ $\newline
        We use Stirling's approximation for this proof \\
        With large values of n, $\Theta (\frac{1}{n}) $is very smaller than 1. \\
        So we can write n! as follows: \\
        \begin{equation}
            \begin{aligned}
            \lg (n !) & \approx \lg \left(\sqrt{2 \pi n}\left(\frac{n}{e}\right)^n\right) \\
            & =\lg (\sqrt{2 \pi n})+\lg \left(\frac{n}{e}\right)^n \\
            & =\lg \sqrt{2 \pi}+\lg \sqrt{n}+n \lg \left(\frac{n}{e}\right) \\
            & =\lg \sqrt{2 \pi}+\frac{1}{2} \lg n+n \lg n-n \lg e \\
            & =\Theta(1)+\Theta(\lg n)+\Theta(n \lg n)-\Theta(n) \\
            & =\Theta(n \lg n)
            \end{aligned}
        \end{equation}
    \end{proof}
    \item Prove equation: \begin{equation} n! = \omega(2^n) \end{equation}
    \begin{proof}
        $ $\newline
        $ $\newline
        We using Stirling's approximation with the limit definition as follows:
        \begin{equation}
            \begin{aligned}
            \lim _{n \rightarrow \infty} \frac{n !}{2^n} & =\lim _{n \rightarrow \infty} \frac{\sqrt{2 \pi n}\left(\frac{n}{e}\right)^n\left(1+\Theta\left(\frac{1}{n}\right)\right)}{2^n} \\
            & =\sqrt{2 \pi} \times \lim _{n \rightarrow \infty} \sqrt{n}\left(\frac{n}{2 e}\right)^n \times \lim _{n \rightarrow \infty}\left(1+\Theta\left(\frac{1}{n}\right)\right) \\
            & =\sqrt{2 \pi} \times \infty \times 1 \\
            & =\infty
            \end{aligned}
        \end{equation}
    \end{proof}
    \item Prove equation: \begin{equation} n! = o(n^n) \end{equation}
    \begin{proof}
        $ $\newline
        $ $\newline
        We using Stirling's approximation with the limit definition as follows:
        \begin{equation}
            \begin{aligned}
            \lim _{n \rightarrow \infty} \frac{n !}{n^n} & =\lim _{n \rightarrow \infty} \frac{\sqrt{2 \pi n}\left(\frac{n}{e}\right)^n\left(1+\Theta\left(\frac{1}{n}\right)\right)}{n^n} \\
            & =\sqrt{2 \pi} \times \lim _{n \rightarrow \infty} \frac{\sqrt{n}}{e^n} \times \lim _{n \rightarrow \infty}\left(1+\Theta\left(\frac{1}{n}\right)\right) \\
            & =\sqrt{2 \pi} \times 0 \times 1 \\
            & =0
            \end{aligned}
        \end{equation}
    \end{proof}
\end{itemize}

\subsection*{Problem 3.2.5}
Which is asymptotically larger: $ \lg(\lg^*n) $ or $ \lg^*(\lg n) $ 
\begin{proof}
    $ $\newline
    $ $\newline
    Follow definition of $ \lg^*n $:
    \begin{equation}
        \lg^*n = min \{i \geq 0: \lg^{(i)} n \leq 1 \}
    \end{equation}
    
    \begin{align*}
        \intertext{We assign} \lg^*n & = a \\
        \intertext{So} \lg(\lg^* n) & = \lg a \\
        \intertext{As we applying logarithm once more thus recucing number of required iterations by 1} \lg^*(\lg n) &c = a - 1 \\
        \intertext{Asymptotically} a - 1 & > \lg a \\
        \lg^*(\lg n) & > \lg(\lg^* n) \\
    \end{align*}
\end{proof}

\section*{Section 2.2}
%
\subsection*{Problem 6}
Blah
\subsection*{Problem 7}
Blah
\subsection*{Problem 10}
Blah

\end{document}