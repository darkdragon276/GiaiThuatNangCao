%This is my super simple Real Analysis Homework template

\documentclass{article}
\usepackage[utf8]{inputenc}
\usepackage[english]{babel}
\usepackage[]{amsthm} %lets us use \begin{proof}
\usepackage[]{amssymb} %gives us the character \varnothing
\usepackage[]{amsmath}

\title{Homework 3}
\author{Chu Hai Nam MSSV: 2370189 \\
        Diep Le Vy MSSV: 2370200 \\
        Huynh Huy Vu MSSV: 2370199 \\
        Nguyen Truong An MSSV: 2370388\\
        Khuat Hoang tri MSSV: 2370127}
\date\today
%This information doesn't actually show up on your document unless you use the maketitle command below

\begin{document}
\maketitle %This command prints the title based on information entered above

\section*{Theorem 34.11}
    Proof that: Clique is NP-complete

\begin{proof}
    Firstly, we all know that for a given graph G = (V, E), if we have set 
    $V$'$ \subseteq V$ of vertices in the Clique, we can check whether for each pair
    $ u, v \in V$'$ $, the edge $ (u, v) $ belongs to G in polynomial time.
    \textbf{This means, $ CLIQUE \in NP $}

    Next, we define the complement of G(V, E) is $ \bar{G} (\bar{V}, \bar{E}) $.
    Call $X$ a clique of G(V, E) if and only if $X$ is an $ IndependentSet $ of
    $ \bar{G} (\bar{V}, \bar{E}) $. \textbf{Therefore,} with $ IndependentSet \leq_p CLIQUE $, 
    if $ IndependentSet \in NP-complete $, then $ CLIQUE \in NP-complete $

    We can create a graph for each statement with 3 variables, e.g $C = (x  \lor \bar{y} \lor z)$, each vertice connects with each other via one edge.
    Then, after connecting 3 statements toghether, the graph will have $3m$ vertices since we have 3 vertices for each statement.
    With each $x_i$, there is only one of the two literals $x_i$ and $ \bar{x_i} $ receives the True value. Therefore, we connect the pairs
    of $x_i$ and $ \bar{x_i} $ toghether. Now, we establish \textbf{the size of clique K = m} since in each statement, only one variable belongs to
    the independent set X

    With each $x_i$, if the assignment of $ \varphi (n, m) $ for $x_i$ value of True, we collect the vertice $x_i$ into set $X$ and vice versa.
    If there are at least 2 literals with value, we will only collect one to make sure that X independent. Therefore, $X$ will have $m$ vertices

    In the opposite side, call $X$ an independent set with the size of $m$. Since the graph have $3m$ vertices and each statement has one the maximum
    of one vertice in $X$, $X$ will have $m$ vertices. In $X$ there is no $x_i$ and $ \bar{x_i} $ due to the structure of the graph (there is no edge
    between them). Therefore, $X$ has $m$ literals and each literal can not be $x_i$ and $ \bar{x_i} $ simulstaneously. We will assign value so that all
    literals in $X$ is True, which makes the $ \varphi (n, m) = True $

    In conclusion, with $ \varphi (n, m) = (x  \lor \bar{y} \lor z) \land (\bar{x}  \lor y \lor z) \land (\bar{x}  \lor y \lor \bar{z}) $,
    there is an assignment that make $ \varphi (n, m) $ True if and only if  G(V, E) has an $ IndependentSet $ with the size of $m$. 
    Therefore,

        $3SAT \leq_p IndependentSet$

        $IndependentSet \leq_p CLIQUE$

        $3SAT \in NP-complete \Rightarrow CLIQUE \in NP-complete$


\end{proof} 

\section*{Theorem 34.12}
    The vertex cover problem is NP-complete.

\begin{proof}
    We will proof this problem is NP-complete by proofing this is both in NP and NP-hard.
    \begin{itemize}
        \item \textbf{NP:} \\
        We will proof that the vertex cover problem is in NP by showing that the certificate is the vertex cover.\\
        Given the graph $G = (V,E)$ and k is interger.\\
        \textbf{Certificate:} A set of vertices, the vertex cover $V'\subseteq V$\\
        \textbf{Verifier:} Check if the set $V'$ is a vertex cover of size k in $G$.\\
        If the set $V'$ is a vertex cover of size k in $G$, then the certificate is accepted.\\
        If the certificate is accepted, then the set $V'$ is a vertex cover of size k in $G$.\\
        \textbf{Proof:} The verifier can check if the set $V'$ is a vertex cover of size k in $G$ by checking if for every edge $(u,v) \in E$, either $u \in V'$ or $v \in V'$.\\
        The running time is polynominal time thus the vertex cover problem is in NP.

        \item \textbf{NP-hard:} \\
        We will proof that the vertex cover problem is NP-hard by reducing the vertex cover problem to the clique problem.\\
        Given the graph $G = (V,E)$ and k is interger.\\
        \textbf{Reduction:} Construct a new graph $G' = (V',E')$ where $V' = V$ and $E' = \{(u,v) | (u,v) \notin E\}$.\\
        \textbf{Claim:} $G$ has a vertex cover of size k if and only if $G'$ has a clique of size $|V| - k$.\\
        \textbf{Proof:} \\
        \begin{itemize}
            \item If $G$ has a vertex cover of size k, then $G'$ has a clique of size $|V| - k$.\\
            Let $V'$ be the vertex cover of size k in $G$.\\
            Let $V'' = V - V'$, then $|V''| = |V| - k$.\\
            Since $V'$ is the vertex cover of size k in $G$, then for every edge $(u,v) \in E$, either $u \in V'$ or $v \in V'$.\\
            Thus, for every edge $(u,v) \notin E$, both $u \in V''$ and $v \in V''$.\\
            Therefore, $V''$ is a clique of size $|V| - k$ in $G'$.
            \item If $G'$ has a clique of size $|V| - k$, then $G$ has a vertex cover of size k.\\
            Let $V''$ be the clique of size $|V| - k$ in $G'$.\\
            Let $V' = V - V''$, then $|V'| = k$.\\
            Since $V''$ is the clique of size $|V| - k$ in $G'$, then for every edge $(u,v) \notin E$, both $u \in V''$ and $v \in V''$.\\
            Thus, for every edge $(u,v) \in E$, either $u \in V'$ or $v \in V'$.\\
            In other words, for every edge $(u,v) \in E$, at least one of $u$ and $v$ is in $V'$.\\
            $V-V'$ is a clique of size $|V| - |V'| = k$ in $G$.\\
            Therefore, $V'$ is the vertex cover of size k in $G$.
        \end{itemize}
    \end{itemize}
    
    Since the vertex cover problem is both in NP and NP-hard, the vertex cover problem is NP-complete.

\end{proof}

\end {document}